\documentclass[xcolor={dvipsnames}]{beamer}

\usetheme{metropolis}

\usepackage{graphicx}
\usepackage{listings}

\author{Joel Berkeley}
\title{Machine learning}
\institute{Verimatrix Bristol}
\date{December 2017}

\setbeamercolor*{structure}{
  bg=OliveGreen!20,
  fg=OliveGreen
}
\setbeamercolor*{palette primary}{
  use=structure,
  fg=OliveGreen,
  bg=structure.fg
}
\setbeamercolor{section in toc}{
  fg=black,
  bg=white
}
\setbeamercolor{alerted text}{
  use=structure,
  fg=structure.fg!50!black!80!black
}
\setbeamercolor{titlelike}{
  parent=palette primary,
  fg=structure.fg!50!black
}
\setbeamercolor{frametitle}{
  bg=gray!10!white,
  fg=OliveGreen
}
\setbeamercolor*{titlelike}{
  parent=palette primary
}

%
%
% Document-specific preamble
%
%

\newtheorem{nofreelunch}{No Free Lunch Theorem}

%
%
% The document
%
%

\begin{document}

\begin{frame}
  \maketitle
\end{frame}

\begin{frame}
  \frametitle{Overview}
  \begin{itemize}
    \item<+-> Why machine learning?
    \item<+-> History
    \item<+-> Foundations \& literature
    \item<+-> Task, performance, experience
    \item<+-> Models
    \item<+-> Deep learning
    \item<+-> Tools \& technologies
    \item<+-> Applications
    \item<+-> AI
  \end{itemize}
\end{frame}

\begin{frame}
  \frametitle{Why machine learning?}
\end{frame}

\begin{frame}
  \frametitle{History}
\end{frame}

\begin{frame}
  \frametitle{Theory prerequisites}
  \begin{itemize}
    \item<+-> Graph theory
    \item<+-> Numerical computation
    \item<+-> Information theory
    \item<+-> Probability theory
  \end{itemize}
\end{frame}

\begin{frame}
  \frametitle{Probability}
  From \ldots\
  \begin{itemize}
    \item<+-> inherent stochasticity e.g. quantum mechanics, chaos
    \item<+-> incomplete observability e.g. hidden variables
    \item<+-> incomplete modelling e.g. discretising continuous data
  \end{itemize}
  Machine learning is essentially a specific application of probability theory
\end{frame}

\begin{frame}
  \frametitle{Task, performance, experience}
  \begin{itemize}
    \item<+-> Task
    \item<+-> Performance
    \item<+-> Experience
      \begin{itemize}
        \item supervised learning
        \item unsupervised learning
        \item ???
      \end{itemize}
  \end{itemize}
\end{frame}

\begin{frame}
  \frametitle{Models}
  \uncover<+>{
    \begin{itemize}
      \item<+-> logistic regression \ldots\ $f: \mathbb R^n \rightarrow \mathbb R $
      \item<+-> $k$-nearest neighbours
      \item<+-> $k$-means clustering
      \item<+-> SVM
      \item<+-> Decision trees \& random forests
      \item<+-> PCA
    \end{itemize}
  }
  \visible<+>{
    \begin{nofreelunch}
      any two optimization algorithms are equivalent when their performance is averaged across all possible problems
    \end{nofreelunch}
    therefore \ldots\
      \begin{itemize}
        \item no ideal universal model
        \item optimize models for specific tasks
      \end{itemize}
  }
\end{frame}

\section{Deep learning}

\begin{frame}
  \frametitle{Usages}
  \begin{itemize}
    \item<+-> Learn the representation
    \item<+-> Many layers
    \item<+-> Hidden layers
  \end{itemize}
\end{frame}

\begin{frame}
  \frametitle{Tools \& technologies}
  \begin{itemize}
    \item<+-> Python \ldots\ Scikit-learn
    \item<+-> Deep learning
      \begin{itemize}
        \item TensorFlow
        \item Caffe (computer vision)
        \item MxNet
        \item Torch (Lua)
        \item DL4J (JVM)
      \end{itemize}
  \end{itemize}
\end{frame}

\begin{frame}
  \frametitle{Applications}
  \begin{itemize}
    \item<+-> Natural langauge processing \& translation
    \item<+-> Computer vision \ldots\ self-driving cars \& medical diagnosis
    \item<+-> Music
    \item<+-> Denoising
  \end{itemize}
\end{frame}

\begin{frame}
  \frametitle{AI}
  \begin{itemize}
    \item Milestones
    \item Artificial conciousness?
  \end{itemize}
\end{frame}

\begin{frame}
  \frametitle{Summary}
  \begin{itemize}
    \item Why machine learning?
    \item History
    \item Foundations \& literature
    \item Task, performance, experience
    \item Models
    \item Deep learning
    \item Tools \& technologies
    \item Applications
    \item AI
  \end{itemize}
\end{frame}

\end{document}
